%\documentclass{beamer}
\documentclass[10pt]{beamer}

\usepackage{amsmath,amssymb,enumerate,calc,color,ifthen,capt-of,booktabs,graphicx,listings,algorithm2e,palatino,amsbsy,subfigure}

\usefonttheme{serif}
 
\definecolor{light-gray}{gray}{0.95}
\lstset{basicstyle=\fontsize{7}{8}\selectfont\ttfamily,
        numbers=left,
        numberstyle=\fontsize{5}{6}\selectfont\ttfamily,
        numbersep=5pt,                  
        %backgroundcolor=\color{light-gray},
        frame=single, 
        %rulesepcolor=\color{red},
        keywordstyle=\color[rgb]{0,0,1},
        commentstyle=\color[rgb]{0.133,0.545,0.133},
        stringstyle=\color[rgb]{0.627,0.126,0.941},
        captionpos=b,
        title=\lstname,
        showstringspaces=false
       }


%---TITLE AND AUTHOR INFORMATION
\title % (optional, use only with long paper titles)
[C++ and Fortran]{Introduction to C++ and Modern Fortran}
\author[Roberts]{Prof.~Jeremy Roberts}
% \institute[22.213] % (optional, but mostly needed)
%  {}
% - Keep it simple, no one is interested in your street address.
\date
%[CFP 2003] % (optional, should be abbreviation of conference name)
{Fall 2017}


\begin{document}

% TITLE PAGE
\begin{frame}[plain]
  \titlepage
\end{frame}

% TABLE OF CONTENTS
\begin{frame}{Outline}
  \tableofcontents
  % You might wish to add the option [pausesections]
\end{frame}

%==============================================================================%

\section{C++ and Fortran Overview}

%------------------------------------------------------------------------------%
\begin{frame}{Hello World}

\begin{columns}[c]
  \begin{column}{0.5\textwidth}
    \lstinputlisting[language=C++]{hello_world.cc}
  \end{column}
  \begin{column}{0.5\textwidth}
    \lstinputlisting[language=Fortran]{hello_world.f90}
  \end{column}
\end{columns}

\end{frame}

%------------------------------------------------------------------------------%
\begin{frame}{Compiling Your First Program}

For C++, use (in the command line)
\begin{equation*}
 \overbrace{\tt g\!+\!+}^{\text{compiler}}~
   \underbrace{\tt{hello\_world.cc}}_{\text{file to compile}}~
     \overbrace{\tt{-o}}^{\text{output as}}~
       \underbrace{\tt{hello\_world}}_{\text{this executable}} 
\end{equation*}
\vfill 

For Fortran, use
\begin{equation*}
 \overbrace{\tt gfortran}^{\text{compiler}}~
   \underbrace{\tt{hello\_world.f90}}_{\text{file to compile}}~
     \overbrace{\tt{-o}}^{\text{output as}}~
       \underbrace{\tt{hello\_world}}_{\text{this executable}} 
\end{equation*}

\vfill 
Use {\tt sudo apt-get install g++ gfortran} to get them.
{\bf Now try them!}

\end{frame}

%------------------------------------------------------------------------------%
\begin{frame}{Compiler Options}

{\tt g++} and {\tt gfortran} are part of the GNU compiler set and share 
several key compiler options that may (or may not) work with compilers from
other vendors; these include:
\begin{itemize}
 \item {\tt -Wall} -- warn us of anything unexpected but make the executable
 \item {\tt -Werror} -- turn any warning into an error
 \item {\tt -O} -- (that's an ``Oh'') use optimization (or {\tt -ON} for $N=0,1,2,3$ for 
       various levels of optimization)
 \item {\tt -g} -- produce debugging information
 \item {\tt -pg} -- produce profiling information
\end{itemize}


\end{frame}

\begin{frame}{Declaring Variables}
\begin{columns}[c]
  \begin{column}{0.5\textwidth}
    \lstinputlisting[language=C++]{declaring.cc}
  \end{column}
  \begin{column}{0.5\textwidth}
    \lstinputlisting[language=Fortran]{declaring.f90}
  \end{column}
\end{columns}
\end{frame}

\begin{frame}{Simple Math}
\begin{columns}[c]
  \begin{column}{0.5\textwidth}
    \lstinputlisting[language=C++]{simple_math.cc}
  \end{column}
  \begin{column}{0.5\textwidth}
    \lstinputlisting[language=Fortran]{simple_math.f90}
  \end{column}
\end{columns}
\end{frame}

\begin{frame}{Control of Program Flow -- If's}
\begin{columns}[c]
  \begin{column}{0.5\textwidth}
    \lstinputlisting[language=C++]{control.cc}
  \end{column}
  \begin{column}{0.5\textwidth}
    \lstinputlisting[language=Fortran]{control.f90}
  \end{column}
\end{columns}
\end{frame}

\begin{frame}{Control of Program Flow -- Switches}
\begin{columns}[c]
  \begin{column}{0.5\textwidth}
    \lstinputlisting[language=C++]{control2.cc}
  \end{column}
  \begin{column}{0.5\textwidth}
    \lstinputlisting[language=Fortran]{control2.f90}
  \end{column}
\end{columns}
\end{frame}

\begin{frame}{Loops}
\begin{columns}[c]
  \begin{column}{0.5\textwidth}
    \lstinputlisting[language=C++]{loops.cc}
  \end{column}
  \begin{column}{0.5\textwidth}
    \lstinputlisting[language=Fortran]{loops.f90}
  \end{column}
\end{columns}
\end{frame}

\begin{frame}{Functions}
\begin{columns}[c]
  \begin{column}{0.5\textwidth}
    \lstinputlisting[language=C++]{functions.cc}
  \end{column}
  \begin{column}{0.5\textwidth}
    \lstinputlisting[language=Fortran]{functions.f90}
  \end{column}
\end{columns}
\end{frame}

%------------------------------------------------------------------------------%
\begin{frame}{Eclipse with C++ and Fortran}

\begin{itemize}
 \item Go to \url{eclipse.org} and download the Eclipse installer
 \item Install Eclipse for Parallel Application Developers
\end{itemize}

\end{frame}


%------------------------------------------------------------------------------%
\begin{frame}{Command Line Arguments}
\begin{columns}[c]
  \begin{column}{0.5\textwidth}
    \lstinputlisting[language=C++]{cl.cc}
  \end{column}
  \begin{column}{0.5\textwidth}
    \lstinputlisting[language=Fortran]{cl.f90}
  \end{column}
\end{columns}
\end{frame}


%------------------------------------------------------------------------------%
\begin{frame}{File I/O}
\begin{columns}[c]
  \begin{column}{0.5\textwidth}
    \lstinputlisting[language=C++]{file_io.cc}
  \end{column}
  \begin{column}{0.5\textwidth}
    \lstinputlisting[language=Fortran]{file_io.f90}
  \end{column}
\end{columns}
\end{frame}

%------------------------------------------------------------------------------%
\begin{frame}{File I/O}
\begin{columns}[c]
  \begin{column}{0.5\textwidth}
    \lstinputlisting[language=C++]{file_io.cc}
  \end{column}
  \begin{column}{0.5\textwidth}
    \lstinputlisting[language=Fortran]{file_io.f90}
  \end{column}
\end{columns}
\end{frame}

%==============================================================================%
\section{C++ Arrays}
%==============================================================================%

%------------------------------------------------------------------------------%
\begin{frame}[allowframebreaks]{C++ Arrays - Main Program}
 \lstinputlisting[language=C++]{basic_arrays.cc }
\end{frame}

%------------------------------------------------------------------------------%
\begin{frame}[allowframebreaks]{C++ Arrays - Function Header}
 \lstinputlisting[language=C++]{basic_arrays_functions.hh}
\end{frame}

%------------------------------------------------------------------------------%
\begin{frame}[allowframebreaks]{C++ Arrays - Function Definitions}
 \lstinputlisting[language=C++]{basic_arrays_functions.cc}
\end{frame}

%------------------------------------------------------------------------------%
\begin{frame}{Compiling}

 Go ahead, try {\tt g++ basic\_array.cc}.

 \pause
 
 \vfill
 The error is a {\bf linking error}.  
 
 \pause 
 
 \vfill
 Comment out the {\tt \#include "basic\_arrays\_functions.hh"} line 
 in {\tt basic\_array.cc} and try compiling again.
 
 \pause 
 \vfill
 The error is a {\bf syntactical error}---all names, including 
 functions, need to be defined before use.
 
 \pause 
 \vfill
 The right way (after uncommenting the {\tt .hh} file): \\
 
 {\tt g++ basic\_array\_functions.cc basic\_array.cc}\\
 
 (where the order matters!)
 
 
\end{frame}

%------------------------------------------------------------------------------%
\begin{frame}{Compiling with {\tt make}}
  \lstinputlisting[language=make]{make_basic_arrays}
\end{frame}

%------------------------------------------------------------------------------%
\begin{frame}[allowframebreaks]{Fortran Arrays - Main Program}
 \lstinputlisting[language=Fortran]{basic_arrays.f90}
\end{frame}

%------------------------------------------------------------------------------%
\begin{frame}[allowframebreaks]{Fortran Arrays - Functions Module}
 \lstinputlisting[language=Fortran]{basic_arrays_module.f90}
\end{frame}

\end{document}
